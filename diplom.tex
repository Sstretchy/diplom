% !TeX encoding = utf-8
\documentclass[a4paper,12pt]{diplom}
\inputencoding{utf8}
\usepackage{paratype}
\DeclareMathSizes{12}{13.4}{11}{10}

\usepackage[left=3cm,right=2cm,top=2cm,bottom=2cm]{geometry} % Размеры полей
\usepackage[onehalfspacing]{setspace} % Полуторный интервал
%\renewcommand{\baselinestretch}{1.25} % Полуторный интервал
\usepackage{indentfirst} % Абзацный отступ в начале разделов
\setlength{\parindent}{1.25cm} % Величина абзацного отступа

\usepackage[pdftex]{graphicx} % Для вставки изображений
\usepackage{array} % Для таблиц
\usepackage{booktabs} % Для красивых таблиц 
\usepackage{tikz} % Рисунки с помощью TikZ
\usepackage[linesnumbered,lined,ruled]{algorithm2e} % Для оформления псевдокода
%\usepackage{algorithm} % Альтернатива оформления псевдокода
%\usepackage{algpseudocode} % Альтернатива оформления псевдокода
\usepackage{listings} % Оформление листингов программ
\usepackage{icomma} % Удаляем тонкий пробел после запятой в мат. режиме

% Если на нумерованную формулу нет ссылки в тексте,
\mathtoolsset{showonlyrefs} % то она становится ненумерованной

% microtype улучшает распределение символов в строке
\usepackage{microtype}  % Можно отключить, если возникают ошибки компиляции

% Формируем PDF с полноценными перекрестными ссылками
\usepackage[unicode, pdfborder={0 0 0}, pdfstartview=FitV]{hyperref}

% Часто используемые макросы
\newcommand{\N}{\mathbb{N}}  % Множество натуральных чисел
\newcommand{\Z}{\mathbb{Z}}  % Множество целых чисел
\newcommand{\R}{\mathbb{R}}  % Множество действительных чисел
\DeclareMathOperator{\sgn}{sgn} % Знак числа
\DeclareMathOperator{\M}{\mathsf{M}} % Матожидание
\newcommand{\from}{\colon} % Двоеточие в определении функции. Пример: $f \from \R \to \N$.
% Заменяем англоязычные обозначения на русские
\renewcommand{\le}{\leqslant}
\renewcommand{\leq}{\leqslant}
\renewcommand{\ge}{\geqslant}
\renewcommand{\geq}{\geqslant}
\renewcommand{\emptyset}{\varnothing}
\renewcommand{\epsilon}{\varepsilon}

%%%%%%%%%%%%%%%%%%%%%%%%%%
% Конец преамбулы
%%%%%%%%%%%%%%%%%%%%%%%%%%

\begin{document}

% Содержимое титульного листа

%\LetterHead{Минобр...}
\Kafedra{Кафедра информационных и сетевых технологий}

% Зав. кафедрой
\ZavKaf{Заведующий кафедрой,\\ к.\,ф.-м.\,н.}{Д.\,Ю.~Чалый}
% Если это курсовая работа и виза зав. каф. не нужна, раскомментируйте следующую строку
%\Kursovaya

% Вид работы: Курсовая работа, Выпускная квалификационная работа, 
\DocumentType{\large Выпускная квалификационная работа}

% Название дипломной работы
%\Title{\begin{Large}\bfseries Название дипломной работы\\ не помещающееся в одну строку\end{Large}}
\Title{\Large\bfseries Разработка клиентской части системы для автоматизации процесса рекрутирования сотрудников}

% Направление подготовки
\Napr{по направлению\\ 09.03.03 Прикладная информатика}

% Руководитель
\Chief{Научный руководитель\\ стар. преподаватель}{Н.\,В.~Легков}

% Автор
\Author{Студент группы ПИЭ-41БО}{О.\,С.~Гаршина}

%\City{Ярославль}
%\Year{2019}

% Создаем титульный лист
\maketitle

\chapter*{Реферат}

Объем \total{page} с., \total{chapternum} гл., \total{fignum} рис.,
\total{tablenum} табл., \total{bibnum} источников, \total{appnum} прил.

\medskip

Ключевые слова и выражения: \textbf{react, рекрутер, автоматизация, HR, резюме, front-end, JavaScript}

\medskip

Целью данной работы является разработка клиентской части приложения - HR-CV Portal, который оптимизирует работу рекрутеров при создании резюме.

В работе проведён анализ потребностей клиента. Также сформированы требования
к приложению, определены технологии для разработки, отвечающие поставленным требованиям. В
результате работы был получен опыт в сфере front-end разработки, а конечный продукт передан
клиенту для использования и получил положительные отзывы и обратную связь для наращивания функционала в дальнейшем.

\medskip

\tableofcontents[Содержание]

\chapternonum{Введение}

Несмотря на то, что мы живем в век технологий и автоматизации есть еще много аспектов, 
требующих алгоритмов, которые не сможет имитировать машина. Такие вещи обычно требуют 
психологических навыков, индивидуального подхода и нажитого опыта. 

В дипломной работе рассматриваются проблемы рекрутеров компании
 Akvelon при бюрократический деятельности, 
а именно проблемы при работе с огромным колчеством резюме, которые надо создавать с нуля, редактировать и
поддерживать в актуальном состоянии.

Заполнение резюме формата комапнии Akvelon
может занимать у сотрудника от часа до четырех часов. Как показал опрос клиента, наибольшей проблемой является время, потраченное
на копирование информации из одного места в другое, орфографические ошибки кандидатов, правки 
съехавшей разметки в word-документе.

В связи с этим, было поставлена задача создать web-приложение, которое бы являлось централизованным хранилищем
 всех резюме компании и цель которого — сделать процесс заполнения данного документа менее рутинным и медленным.

\chapter{Теория}

\chapter{О задаче}

\section{Постановка задачи}

Требуется создать web-приложение, которое упрощает создание и обновление резюме кандидатами и работниками компании Akvelon,
а также решает многие проблемы рекрутеров, оптимизируя их работу и тем самым сокращая время,
проведенное над редактированием документов.

\section{Требуемый функционал}

\begin{enumerate}
  \item Возможность создания, копирования, редактирования и архивирования резюме;
	\item Наличие базы данных, в которой бы хранились названия компаний, институтов;
	проектов, навыков, персональных результатов и сфер ответственности;
	\item Автоматическое заполнение перечисленных данных в поля резюме - всплывающие подсказки и поиск по ним;
	\item Возможность пополнения этой базы данных как обычными пользователями так и администраторами сайта;
	\item Модерация добавленных данных администраторами в один клик;
	\item Подобие папок с проектами, на которые можно назначить кандидатов и производить поиск по имени и позиции;
	\item Скачивание резюме в формате .docx, стилизованное под стандартное резюме Аквелона;
	\item Заполнение общей информации о кандидате с помощью подсказок с логическими выделенными словосочетаниями;
	 которые превращаются в подобие шаблона при их выборе. Подсказки должны предлагаться в случайном порядке, чтобы повысить уникальность текста в резюме;
	\item Пользователь приложения должен иметь возможность указать свою роль на проекте, для которого создается резюме;
	 Смена этой роли должна вызвать автоматическую пересортировку данных, 
	 чтобы наиболее актуальные для позиции умения находились выше остальных;
	 \item Сайт нужно создать в стиле Аквелона, придерживаясь дизайна других сервисов данной 
	 компании;
	 \item Возможность дать другим пользователям права модератора;
	 \item Блокировка и удаление пользователя;
	 \item Всплывающие уведомления об ошибках и прочей информации для пользователя;
	 \item Интерфейс для отслеживания прогресса работы приложения;
\end{enumerate}

\section{Используемые программные средства}

Исходя из того, что требуется написать клиентскую часть приложения, для разработки были выбраны следующие программные средства:

\begin{itemize}
  \item VSCode для разработки и отладки приложения;
  \item JavaScript в качестве основного языка программирования;
  \item GitLab для управления репозиторием кода для Git;
  \item MobX для управления состоянием приложения;
  \item Axios для взаимодействия с API;
  \item React.js для создания интерфейса;
  \item Material UI для создания единого стиля компонентов;
  \item Less для корректировок стиля Matreial и для создания собственного.
  \item Jest и Enzyme для написания unit-тестов.
\end{itemize}

\chapter{Решение задачи}

\section{Создание базовой архитектуры}
В компании мне предоставили готовый шаблон со структурой, где уже подключен Webpack, настроено несколько правил ESLint для поддержания кода чистым и более приятным глазу.
Для начала разработки была релизована следующая структура проекта в директории src:

\renewcommand*\DTstyle{\ttfamily\textcolor{black}}
\dirtree{%
  .1 /.
  .2 components.
  .2 containers.
  .2 services.
  .3 action.notify.
  .3 toast.notify.
  .3 stores.
  .3 request.services.
  .3 validator.
  .2 app.js.
  .2 app.less.
  .2 constants.js.
  .2 index.html.
  .2 muitheme.js.
  .2 router.js.
  .2 variables.less.
}

\begin{itemize}
  \item Папка components предназначена для react-компонентов для многоразового использования, непривязанных к какому-либо контексту, желательно максимально абстрактных.
  \item Containers - каталог для логически разделенных папок, содержащих в себе компоненты конкретных страниц.
  \item Services - папка для сервисов, которые отвечают за реализацию кода, независимого от внешнего окружения. В данном приложении понадобились сервисы для логики полос прогрузки данных, появления уведомлений, взаимодействия с API, валидации, и действий с observable-состаяниями MobX-а.
  \item index.html - точка входа приложения, в нем описываются подключения стилей и скрипта для рендера.
  \item index.js указывает, в какую область html документа будет проецироваться DOM-дерево и рендерит app.js.
  \item app.js содержит компоненты-провайдеры, отвечающие за авторизацию, инициализацию MobX stores, стилей-muitheme и перенаправления на страницы.
  \item app.less - в этой файле написаны общие стили, которые используются практически во всех компонентах.
  \item constants.js - переменные, которые используются в разных местах программы по типу предложений, коэффициентов, регулярных выражений.
  \item muitheme.js - файл, позволяющий задать конфигурации Material UI стилей.
  \item router.js - компонент-маршрутизатор, определяет какой обработчик надо вызвать для конкретного маршрута.
  \item variables.less - содержит палитру именных основных цветов сайта. Файл служит для удобства, чтобы было проще ориентироваться на название переменной, а не на HEX или RGB коды.
\end{itemize}

\section{Регистрация и вход на сервис}

Клиент поставил условие, что страницы, связанные с авторизацией должны быть выполнены в таком же стиле, что и сайт компании для
оценки рабочего времени, написанный на Vue.js. Но должна быть возможность входа с любой почтой, а не с доменным именем.

\begin{figure}[!ht]
	\centering
	\includegraphics[width=1\textwidth]{resources/ets.png}
	\caption{Страница входа на ets.akvelon.com}
	\label{fig:1}
\end{figure}


\section{Компиляция умных контрактов}

\section{Развертывание умных контрактов в локальной сети}

\section{Развертывание умных контрактов в удаленных сетях}

\section{Отладка умных контрактов}

\section{Визуальный интерфейс для взаимодействия с развернутыми умными контрактами}

\chapter{Результаты решения задачи}

В результате решения задачи было получено расширение для VSCode

\chapternonum{Заключение}

% Если нужно, меняем название Литература
\renewcommand{\bibname}{Список литературы} 
\begin{thebibliography}{9}
% Если нужно сделать N. вместо [N] 
% \makeatletter\renewcommand{\@biblabel}[1]{#1.}\makeatother

\bibitem{VSCode}
Visual Studio Code - Code Editing. Redefined
URL: https://code.visualstudio.com
(дата доступа: 09.06.2020)

\end{thebibliography}

\end{document}
