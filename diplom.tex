% !TeX encoding = utf-8
\documentclass[a4paper,12pt]{diplom}
\inputencoding{utf8}
\usepackage{paratype}
\DeclareMathSizes{12}{13.4}{11}{10}

\usepackage[left=3cm,right=2cm,top=2cm,bottom=2cm]{geometry} % Размеры полей
\usepackage[onehalfspacing]{setspace} % Полуторный интервал
%\renewcommand{\baselinestretch}{1.25} % Полуторный интервал
\usepackage{indentfirst} % Абзацный отступ в начале разделов
\setlength{\parindent}{1.25cm} % Величина абзацного отступа

\usepackage[pdftex]{graphicx} % Для вставки изображений
\usepackage{array} % Для таблиц
\usepackage{booktabs} % Для красивых таблиц 
\usepackage{tikz} % Рисунки с помощью TikZ
\usepackage[linesnumbered,lined,ruled]{algorithm2e} % Для оформления псевдокода
%\usepackage{algorithm} % Альтернатива оформления псевдокода
%\usepackage{algpseudocode} % Альтернатива оформления псевдокода
\usepackage{listings} % Оформление листингов программ
\usepackage{icomma} % Удаляем тонкий пробел после запятой в мат. режиме

% Если на нумерованную формулу нет ссылки в тексте,
\mathtoolsset{showonlyrefs} % то она становится ненумерованной

% microtype улучшает распределение символов в строке
\usepackage{microtype}  % Можно отключить, если возникают ошибки компиляции

% Формируем PDF с полноценными перекрестными ссылками
\usepackage[unicode, pdfborder={0 0 0}, pdfstartview=FitV]{hyperref}

% Часто используемые макросы
\newcommand{\N}{\mathbb{N}}  % Множество натуральных чисел
\newcommand{\Z}{\mathbb{Z}}  % Множество целых чисел
\newcommand{\R}{\mathbb{R}}  % Множество действительных чисел
\DeclareMathOperator{\sgn}{sgn} % Знак числа
\DeclareMathOperator{\M}{\mathsf{M}} % Матожидание
\newcommand{\from}{\colon} % Двоеточие в определении функции. Пример: $f \from \R \to \N$.
% Заменяем англоязычные обозначения на русские
\renewcommand{\le}{\leqslant}
\renewcommand{\leq}{\leqslant}
\renewcommand{\ge}{\geqslant}
\renewcommand{\geq}{\geqslant}
\renewcommand{\emptyset}{\varnothing}
\renewcommand{\epsilon}{\varepsilon}

%%%%%%%%%%%%%%%%%%%%%%%%%%
% Конец преамбулы
%%%%%%%%%%%%%%%%%%%%%%%%%%

\begin{document}

% Содержимое титульного листа

%\LetterHead{Минобр...}
\Kafedra{Кафедра информационных и сетевых технологий}

% Зав. кафедрой
\ZavKaf{Заведующий кафедрой,\\ к.\,ф.-м.\,н.}{Д.\,Ю.~Чалый}
% Если это курсовая работа и виза зав. каф. не нужна, раскомментируйте следующую строку
%\Kursovaya

% Вид работы: Курсовая работа, Выпускная квалификационная работа, 
\DocumentType{\large Выпускная квалификационная работа}

% Название дипломной работы
%\Title{\begin{Large}\bfseries Название дипломной работы\\ не помещающееся в одну строку\end{Large}}
\Title{\Large\bfseries Разработка клиентской части системы для автоматизации процесса рекрутирования сотрудников}

% Направление подготовки
\Napr{по направлению\\ 09.03.03 Прикладная информатика}

% Руководитель
\Chief{Научный руководитель\\ стар. преподаватель}{Н.\,В.~Легков}

% Автор
\Author{Студент группы ПИЭ-41БО}{О.\,С.~Гаршина}

%\City{Ярославль}
%\Year{2019}

% Создаем титульный лист
\maketitle

\chapter*{Реферат}

Объем \total{page} с., \total{chapternum} гл., \total{fignum} рис.,
\total{tablenum} табл., \total{bibnum} источников, \total{appnum} прил.

\medskip

Ключевые слова: \textbf{react, рекрутер, автоматизация, HR, резюме, front-end, JavaScript}

Текст реферата должен отражать объект исследования, цель работы, результаты работы, область применения, степень внедрения или рекомендации по~внедрению.

\medskip

\tableofcontents[Содержание]

\chapternonum{Введение}

Во введении обосновывается актуальность выбранной темы,
описываются объект и~предмет исследования, цели и~задачи, методы исследования и~приводится краткое описание структуры работы.

\chapter{Теория}

\chapter{О задаче}

\section{Постановка задачи}

Требуется создать расширение для интегрированной среды разработки Visual Studio Code, позволяющее создавать,
собирать, отлаживать, разворачивать и взаимодействовать с умными контрактами в сети Ethereum

\section{Требуемый функционал}

\begin{enumerate}
  \item Подсветка синтаксиса умных контрактов и автоматическое дополнение ключевых слов по сокращениям
  \item Компиляция умных контрактов
  \item Развертывание умных контрактов в локальной сети
  \item Развертывание умных контрактов в удаленных сетях
  \item Отладка умных контрактов
  \item Визуальный интерфейс для взаимодействия с развернутыми умными контрактами
\end{enumerate}

\section{Используемые программные средства}

Основываясь на том, что расширение пишется для IDE VSCode, для разработки были выбраны следующие программные средства:

\begin{itemize}
  \item VSCode для разработки и отладки расширения
  \item TypeScript в качестве основного языка программирования
  \item JavaScript в качестве языка программирования при создании графического интерфейса взаимодействия в методами умного контракта
  \item Truffle для компиляции, отправки кода контракта в цепочку блоков
  \item Ganache для создания локальной сети Ethereum
  \item Drizzle в качестве прослойки между актуальным состоянием развернутого контракта в сети и его отображением в внутреннем состоянии JavaScript
  \item React.js для создания визуального интерфейса взаимодействия с умными контрактами 
\end{itemize}

\chapter{Решение задачи}

\section{Подготовка окружения расширения}

Компания Microsoft советует использовать утилиту yo в связке с generator-code. yo отвечает отвечает за запуск пошагового руководства
для первоначальной настройки структуры, а generator-code предоставляет сами шаги настройки. Этим инструментарием было решено и воспользоваться.

После выполнения команды yo code и прохождения по предлагаемым шагам настройки получаем следующую структуру проекта:

\renewcommand*\DTstyle{\ttfamily\textcolor{black}}
\dirtree{%
  .1 /.
  .2 .vscode.
  .3 launch.json.
  .3 tasks.json.
  .2 .gitignore.
  .2 README.md.
  .2 src.
  .3 extension.ts.
  .2 package.json.
  .2 tsconfig.json.
}

\begin{itemize}
  \item launch.json отвечает за конфигурацию запуска разрабатываемого расширения в режиме отладки
  \item tasks.json отвечает за описание задач, вызываемых из launch.json
  \item extension.ts является точкой запуск расширения, в нем принято регистрировать обработчики события,
  вызываемых пользователем расширения
  \item package.json
  \item tsconfig.json
\end{itemize}

\section{Подсветка синтаксиса и авто дополнение по сокращениям}

\section{Компиляция умных контрактов}

\section{Развертывание умных контрактов в локальной сети}

\section{Развертывание умных контрактов в удаленных сетях}

\section{Отладка умных контрактов}

\section{Визуальный интерфейс для взаимодействия с развернутыми умными контрактами}

\chapter{Результаты решения задачи}

В результате решения задачи было получено расширение для VSCode

\chapternonum{Заключение}

% Если нужно, меняем название Литература
\renewcommand{\bibname}{Список литературы} 
\begin{thebibliography}{9}
% Если нужно сделать N. вместо [N] 
% \makeatletter\renewcommand{\@biblabel}[1]{#1.}\makeatother

\bibitem{VSCode}
Visual Studio Code - Code Editing. Redefined
URL: https://code.visualstudio.com
(дата доступа: 09.06.2020)

\end{thebibliography}

\end{document}
